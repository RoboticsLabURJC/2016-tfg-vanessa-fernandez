\chapter{Práctica: Aparcamiento Automático}\label{cap.autoparking}
En este capítulo se describirá el desarrollo de una nueva práctica para el entorno JdeRobot-Academy, que se denomina ``Aparcamiento automático''. En este capítulo se aborda el desarrollo de su infraestructura, su componente académico, así como el evaluador atomático que se ha creado y la solución de referencia realizada.  

\section{Enunciado}
El problema abordado en esta práctica es que un coche autónomo sea capaz de aparcar en una plaza de aparcamiento en línea sin chocar con los coches que están delante y detrás de la plaza libre de aparcamiento. El coche dispondrá de un \acrshort{gps} que le proporciona una estimación de su posición en el entorno en que se encuentra. Este coche tiene también tres sensores láser mediante los cuales podrá obtener información acerca del entorno por el que se mueve. Este vehículo posee un actuador de movimiento que se basa en la velocidad de tracción y la velocidad de giro.\\

El alumno deberá programar el comportamiento de este vehículo autónomo para que pueda aparcar. En la interfaz gráfica del componente académico desarrollado se puede visualizar un mapa local del entorno por el que se mueve el coche, así como la posición de este vehículo en el mapa. En esta interfaz, hay un visor que muestra gráficamente lo que ``ve'' cada láser del entorno. Esta interfaz gráfica facilitará al alumno la resolución de la práctica.\\

El algoritmo propuesto como solución de referencia responde a un control reactivo, donde en cada momento el coche actuará en función de los datos de los sensores o de variables internas. El control reactivo permitirá controlar en todo momento el movimiento del vehículo de forma que pueda responder ante situaciones imprevistas.

\section{Infraestructura}

En este apartado se describirá el entorno que se ha creado para poder realizar la práctica ``Aparcamiento automático''. Se comenzará describiendo el modelo de coche empleado, así como los sensores y actuadores que posee el mismo. Después, se realizará una explicación del entorno simulado por el cual se moverá el coche.

\subsection{Modelo Taxi\_Holo\_Laser}
El robot que se ha empleado en esta práctica es un nuevo modelo de coche basado en el modelo de taxi que se empleó en la práctica ``global\_navigation''. Este modelo de coche se denomina \textit{taxi\_holo\_Laser} y lo podemos ver en Figura~\ref{fig.taxiAutopark}. Posee sensores \acrshort{gps} que le permiten saber cuál es su posición en todo momento; tiene tres sensores láser, uno se sitúa en la parte frontal del coche, otro en la parte trasera, y el último a la derecha del coche. También incorpora motores que le permiten moverse por el escenario de manera adecuada.

\begin{figure}[H]
  \begin{center}
    \includegraphics[width=0.6\textwidth]{figures/Autopark/taxiAutopark.png}
		\caption{Modelo taxi\_holo\_Laser}
		\label{fig.taxiAutopark}
		\end{center}
\end{figure}

En esta práctica se han empleado los \textit{plugins}:

\begin{itemize}
\item \textit{holoCarPose3D}: Los componentes emplean este \textit{plugin} para obtener la posición del coche en tiempo real. 
\item	\textit{holoCarMotors}: El componente académico interactúa con este \textit{plugin}. Este \textit{plugin} permite dotar al componente de velocidad, tanto velocidad de tracción como velocidad de rotación.
\item	\textit{laser}: Este \textit{plugin} será empleado por los componentes para obtener datos de la distancia que hay hasta los obstáculos.
\end{itemize}

\subsubsection{Sensores láser}
En este taxi se han instalado tres sensores láser. Un sensor se ha colocado en la parte frontal del vehículo, otro en la parte trasera, y el último en la parte derecha del coche. Estos sensores serán empleados en el algoritmo de la práctica. Los sensores láser están formados por un array de 180 medidas, al igual que los empleados en el Capítulo~\ref{cap.roomba}. \\


\subsection{Modelo acera} \label{sec.acera}
El objetivo de la práctica es que el coche sea capaz de aparcar de forma autónoma. Para ello es necesario crear un entorno (los modelos en gazebo) donde se moverá el coche. El coche deberá aparcar paralelo a una acera, por lo tanto, se ha creado el modelo \textit{``acera''}. Este modelo se puede ver en la Figura~\ref{fig.acera}.

\begin{figure}[H]
  \begin{center}
    \includegraphics[width=0.7\textwidth]{figures/Autopark/acera.png}
		\caption{Modelo acera}
		\label{fig.acera}
		\end{center}
\end{figure}

\subsection{Modelo carNoMotor}
El coche deberá aparcar entre dos vehículos, y habrá más coches estacionados. Por este motivo se ha creado el modelo \textit{``carNoMotor''}, el cual no posee motores, ya que queremos que esté estacionado. Este modelo se utilizará varias veces al crear el mundo de Gazebo. En la Figura~\ref{fig.carNoMotor}, podemos ver este modelo.

\begin{figure}[H]
  \begin{center}
    \includegraphics[width=0.5\textwidth]{figures/Autopark/carNoMotor.png}
		\caption{Modelo carNoMotor}
		\label{fig.carNoMotor}
		\end{center}
\end{figure}

\subsection{Mundo de Gazebo}
Este mundo estará formado por un modelo de carretera (\textit{road}) que tiene Gazebo, poseerá dos aceras para lo cual emplearemos el modelo \textit{``acera''}, además se emplearán varios coches del modelo \textit{``carNoMotor''} que estarán aparcados; y, por último, se incluirá el modelo del coche (\textit{taxi\_holo\_Laser}), que ejecutará la solución que le indiquemos. Para tener este escenario se ha creado un mundo en Gazebo llamado \textit{``autopark.world''}:

\vspace{20pt}
	\begin{lstlisting}[frame=single]
<?xml version="1.0"?>
<sdf version="1.4">
  <world name="default">

    <scene>
      <grid>true</grid>
    </scene>

    <!-- A global light source -->
    <include>
      <uri>model://sun</uri>
    </include>

    <!-- Ground -->
    <include>
      <uri>model://ground_plane_sincolor</uri>
    </include>
    <include>
      <uri>model://acera</uri>
      <pose>5 9 0 0 0 0</pose>
    </include>
    <include>
      <uri>model://acera</uri>
      <pose>5 -9 0 0 0 0</pose>
    </include>

    <!-- A taxi with lasers-->
    <include>
      <uri>model://taxi_holo_Laser</uri>
      <pose>-7 2.5 0 0 0 0</pose>
    </include>

    <!-- Cars -->
    <include>
      <uri>model://carNoMotor</uri>
      <pose>-20 -3 0 0 0 1.57</pose>
    </include>
    <include>
      <uri>model://carNoMotor</uri>
      <pose>-13.5 -3 0 0 0 1.57</pose>
    </include>
    <include>
      <uri>model://carNoMotor</uri>
      <pose>-7 -3 0 0 0 1.57</pose>
    </include>
    <include>
      <uri>model://carNoMotor</uri>
      <pose>0.5 -3 0 0 0 1.57<pose>
    </include>
    <include>
      <uri>model://carNoMotor</uri>
      <pose>14 -3 0 0 0 1.57</pose>
    </include>
    <include>
      <uri>model://carNoMotor</uri>
      <pose>21.5 -3 0 0 0 1.57</pose>
    </include>
    <include>
      <uri>model://carNoMotor</uri>
      <pose>29 -3 0 0 0 1.57</pose>
    </include>

    <!-- A Road -->
    <road name="my_road_1">
      <width>10</width>
      <point>-25 0 0.02</point>
      <point>35 0 0.02</point>
    </road>

  </world>
</sdf>
	\end{lstlisting}


En la Figura~\ref{fig.Mundo_Autopark} podemos ver cuál es el aspecto que tiene el mundo que hemos creado en Gazebo.

\begin{figure}[H]
  \begin{center}
    \includegraphics[width=0.8\textwidth]{figures/Autopark/Mundo_Autopark.png}
		\caption{Mundo autopark.world en Gazebo}
		\label{fig.Mundo_Autopark}
		\end{center}
\end{figure}

\section{Componente Académico}
El componente académico desarrollado resuelve diversas funcionalidades en la práctica: (a) ofrece una interfaz gráfica al usuario, que le ayuda a depurar su código; (b) ofrece acceso a sensores y actuadores en forma de métodos simples (oculta el \textit{middleware} de comunicaciones); (c) incluye código auxiliar que no es el centro del algoritmo y que ayuda a programar la solución. El componente deja todo listo para que el estudiante sólo tenga que incorporar su código rellenando el método \textit{execute} en el fichero \textit{MyAlgorithm.py}.\\

Este componente ofrece al programador del algoritmo este \acrshort{api} de sensores y actuadores:

\begin{itemize}
\item \textit{pose3d.getX()}: Permite obtener la posición del robot en el eje X.
\item	\textit{pose3d.getY()}: Permite obtener la posición del robot en el eje Y.
\item	\textit{pose3d.getYaw()}: Permite obtener la orientación del robot con respecto al mapa.
\item	\textit{laser.getLaserData()}: Permite obtener los datos del sensor láser, que se compone de 180 pares de valores (0-180º, distancia en milímetros). Esta función se empleará en cada uno de los láseres para obtener sus datos.
\item \textit{motors.sendV()}: Para establecer la velocidad lineal.
\item	\textit{motors.sendW()}: Para establecer la velocidad de giro.
\end{itemize}

Es necesario crear un archivo de configuración en la práctica similar a los de las prácticas anteriores:

\vspace{20pt}
	\begin{lstlisting}[frame=single]
Autopark.Motors.Proxy  = Motors:default -h localhost -p 9999
Autopark.Pose3D.Proxy  = Pose3D:default -h localhost -p 9989
Autopark.Laser1.Proxy  = Laser:default -h localhost -p 8996
Autopark.Laser2.Proxy  = Laser:default -h localhost -p 8997
Autopark.Laser3.Proxy  = Laser:default -h localhost -p 8998

Autopark.Motors.maxV = 250
Autopark.Motors.maxW = 20

	\end{lstlisting}


Los motores emplean el puerto 9999, el Pose3D emplea el puerto 9989, y los sensores láser utilizan los puertos 8996, 8997, 8998. En este archivo se indica también la velocidad lineal máxima y la velocidad angular máxima.\\

En la práctica el comportamiento se divide en varias tareas. Para realizar estas tareas simultáneamente, empleamos dos hilos de ejecución:

\begin{itemize}
\item Hilo de control: Este hilo se encarga de actualizar los datos de los sensores y los actuadores a través de las interfaces \acrshort{ice}. El tiempo de actualización de este hilo es muy importante, ya que este componente establece la velocidad y la dirección del robot en todo momento. Este intervalo de actualización debe ser un periodo de tiempo muy corto. Si fuera muy grande, las decisiones que modifican la trayectoria del robot podrían influir en su comportamiento, haciendo que la trayectoria sea incorrecta. En esta práctica el hilo de control de actuadores y sensores se actualizará cada vez que se actualice la \acrshort{gui}, es decir, cada 50 ms.
\item	Hilo de la interfaz gráfica de usuario (\acrshort{gui}): Se encarga de actualizar la interfaz gráfica. El intervalo de actualización de la interfaz gráfica debe ser corto, puesto que se tiene que mostrar la posición del robot en el mapa que muestra la interfaz en tiempo real. El intervalo es de 50 ms.

\end{itemize}

\subsection{Interfaz gráfica}
La interfaz gráfica de usuario (\acrshort{gui}) se emplea para representar información que pueda ayudar a resolver el algoritmo planteado. Esta interfaz sirve para ejecutar el código que da solución a la práctica. Esta interfaz es programada en PyQt5.\\

Esta \acrshort{gui} (Figura~\ref{fig.GUI_Autopark}) contiene un lienzo en blanco donde se ha pintado parte del mundo de Gazebo, así como la posición del coche. En concreto, hay ciertas partes pintadas en negro, que se corresponden con las aceras y los coches aparcados, es decir, los obstáculos. Por el contrario, las zonas que aparecen en blanco es donde no hay obstáculos, es decir, la carretera sin objetos. El coche aparece representado mediante un rectángulo amarillo con ruedas en negro. Este rectángulo también rota cuando el coche lo hace, de esta forma se puede visualizar adecuadamente su comportamiento. En este pequeño mapa local del escenario podemos ver que aparecen pintadas de verde las aristas de un rectángulo, el cual se corresponde con la posición ideal que debería conseguir el coche al aparcar. También podemos observar unos puntos rojos que aparecen consecutivamente, los cuales se corresponden con la estela del coche, es decir, las posiciones que ha tenido recientemente. Pero todas las posiciones que tiene el coche no se pintan, solamente las últimas. Hemos creado un array de cierto tamaño para representar únicamente unos cuantos puntos y no todos. Es un ``array circular'', se van añadiendo las nuevas posiciones por el final del array y cuando llega a un cierto tamaño vamos borrando la primera posición del array y desplazando el contenido de sus posiciones a una posición anterior. \\

En la zona central del \acrshort{gui} se representa el coche mediante un rectángulo amarillo (con sus ruedas en negro), y los datos que ofrecen los sensores láser. Cada sensor láser se ha pintado de un color diferente para distinguirlos. Si comparamos esta representación con el mundo de Gazebo se podría ver que la representación es idéntica a cómo se ven los sensores láser en Gazebo. Para representar cada uno de los sensores en la posición adecuada se han empleado matrices de rotación y traslación al igual que sucedía en otras prácticas.\\

A la derecha del \acrshort{gui}, tenemos un teleoperador para mover al robot. Este teleoperador controla las velocidades lineal y angular del robot. \\

En la interfaz aparecen dos botones. El botón de debajo del mapa permitirá ejecutar el código de solución a la práctica. También permitirá parar el código de la solución, de forma que el coche se quede parado donde estuviera situado. El botón que aparece debajo de la gráfica con los sensores láser y el teleoperador, permitirá detener al robot si lo estamos teledirigiendo con el teleoperador. \\

\begin{figure}[H]
  \begin{center}
    \includegraphics[width=0.9\textwidth]{figures/Autopark/GUI_Autopark.png}
		\caption{\acrshort{gui} del Autopark}
		\label{fig.GUI_Autopark}
		\end{center}
\end{figure}

\section{Solución de referencia}
En esta sección se describirá cómo aparcan los coches autónomos reales, así como la solución ``ad hoc'' que se ha desarrollado. Se exploran qué algoritmos han diseñado los fabricantes de coches para equipar a sus vehículos con esa capacidad (Volskwagen Tiguan...).

\subsection{Algoritmos empleados por los coches autónomos reales}
Durante la conducción, una de las maniobras más complicadas es el aparcamiento del vehículo, en especial en ciertas zonas donde los huecos disponibles son bastantes escasos. En la actualidad existen distintos avances tecnológicos para facilitar el aparcamiento al conductor.\\


Uno de los primeros prototipos de coche con aparcamiento autónomo fue producido en Francia, hace alrededor de 20 años por el INRIA (Institut National de Recherche en Informatique et en Automatique). Fue el primero en realizar un aparcamiento autónomo en paralelo. Hoy en día, cada vez más fabricantes de vehículos incorporan tecnologías de conducción autónoma. Se ha definido un estándar, es J3016, que establece niveles de conducción autónoma según la capacidad del vehículo.

\begin{itemize}
\item Nivel 0: No hay automatización de la conducción. Las tareas de conducción son realizadas en su totalidad por el conductor.
\item	Nivel 1: Asistencia al conductor. El vehículo tiene algún sistema de automatización de la conducción (control de crucero, autoaparcamiento), ya sea para el control de movimiento longitudinal o el movimiento lateral, pero no ambas cosas al mismo tiempo. El conductor realiza el resto de tareas de conducción, por lo que debe estar siempre atento.
\item	Nivel 2: Automatización parcial. Considera que el conductor ya no tiene que conducir en todo momento y que el coche empieza a ser realmente autónomo, aunque con matices. El vehículo es capaz de actuar de forma independiente dentro de escenarios controlados y en situaciones específicas de conducción. El conductor debe seguir prestando atención a lo que ocurre a su alrededor para evitar posibles riesgos. Un buen ejemplo de Nivel 2 de conducción autónoma pueden ser los modelos BMW Serie 7 o el Mercedes Clase E, capaces de ir solos durante un tiempo o con el sistema de asistente de atascos.
\item	Nivel 3: Automatización condicional. En este nivel, el coche empieza a interactuar con lo que le rodea y es capaz de analizar posibles riesgos externos para evitarlos. Ya no se habla de conductor sino que hablamos de un usuario preparado para intervenir, es decir, el coche ya conduce completamente solo y el conductor es un simple vigilante de que todo funcione correctamente. El coche está preparado para ser conducido de manera habitual en cualquier momento. 
\item Nivel 4: Alta autonomía. En este nivel el sistema cuenta tanto con los sistemas de automatización presentes en el anterior nivel, como con sistemas de detección de objetos y eventos. Además, es capaz de responder ante ellos. El sistema de automatización de la conducción tiene un sistema de respaldo para actuar en caso de fallo del sistema principal y poder conducir hasta una situación de riesgo mínimo. En algunas situaciones es posible que el vehículo no siga conduciendo.
\item	Nivel 5: Autonomía total. Este nivel cuenta con todos los beneficios del sistema de automatización del nivel 4. Sin embargo, la diferencia es que en este caso el vehículo podría seguir conduciendo en todo momento o circunstancia.

\end{itemize}

Ejemplos seminales importantes de conducción autónoma son: el DARPA Grand Challenge y el Urban Challenge. El DARPA Grand Challenge, organizado en 2004 y 2005 en Estados Unidos, fue una carrera de vehículos autónomos que debían recorrer 120 kms por el desierto de Nevada sin intervención humana y disponiendo únicamente de un listado de puntos intermedios entre el principio del circuito y el final. El Urban Challenge organizado en 2007, fue una carrera de vehículos autónomos por zona urbana en la que debían recorrer 96 km en menos de 6 horas. \\

La tecnología que se emplea para el aparcamiento autónomo tiene que tener en cuenta diversos factores para aparcar: espacio disponible, maniobras a realizar y posición. Para conseguir este objetivo los coches emplean sensores que miden la distancia desde el coche hasta los límites de la plaza y los otros coches u obstáculos, tanto en el parachoques trasero como en el parachoques delantero. Estos sensores suelen ser de ultrasonidos, y su número y distribución depende del tamaño y diseño del coche, aunque suelen ser cuatro o cinco por parachoques. En algunas ocasiones se puede emplear un radar como sensor.\\

En los laterales de los paragolpes es necesario que haya más sensores orientados de forma transversal. De esta forma pueden medir la distancia hacia el lateral del coche. Estos sensores permiten identificar si existe un hueco en el que aparcar y si el tamaño del hueco es suficiente para que el coche entre.\\

La firma Bosch lidera el suministro de tecnología para la automoción y tiene a la totalidad de marcas fabricantes de automóviles como clientes. Un sistema que Bosch proporciona a las marcas es la ``cámara de marcha atrás''. Esta cámara está ubicada en la parte posterior del vehículo, y se activa de forma automática cuando el conductor da marcha atrás. Inmediatamente aparecen imágenes en color del entorno en el monitor y las líneas límite de maniobra para evitar roces, así como advertencias acústicas.\\

Otro sistema importante de Bosch es el sistema Multi-cámara, que cuenta con cuatro cámaras con apertura de 190º que captan todo el entorno del vehículo y se reflejan en la pantalla del salpicadero (un ejemplo es Nissan Quasquai). Estas imágenes son imágenes en 3D sin distorsión.\\

Bosch también incorpora un dispositivo llamado ``freno de emergencia en maniobras''. Este dispositivo detecta obstáculos y objetos en movimiento (mediante sensores ultrasonidos) hasta una distancia de cuatro metros y a una velocidad de no más de 10 km/h. Si hay riesgo de colisión avisa al conductor y si éste no reacciona, activa una frenada de emergencia.\\

La nueva generación del Audi A8 es uno de los primeros coches dotados de un sistema de conducción autónoma completa. Cuenta con el innovador sistema que se llama \textit{``Traffic Jam Assist''}, el cual permite llevar a cabo una conducción autónoma. Además, incluye el denominado \textit{``Park Assist''} para aparcar el coche desde fuera del vehículo por medio de una aplicación para el teléfono móvil.\\

El nuevo modelo de Tesla, llamado Model 3, incluye uno de los sistemas de conducción autónoma más avanzados e incorpora el nuevo sistema inteligente \textit{Autopilot} que le permite aparcar sólo, entre otras capacidades. Gracias a las cámaras instaladas en la parte frontal y trasera del coche, es capaz de aparcar adecuadamente y desenvolverse sin problema en gran parte de las condiciones de tráfico sin ayuda de un conductor humano.\\

Las compañías alemanas Daimler y Bosch han presentado el sistema \textit{“Automated Valet Parking”} que combina la tecnología en el propio coche con los sensores presentes en el edificio de aparcamientos para conseguir aparcar el vehículo automáticamente y sin nuestra atención. El concepto es simple: llegas al parking, te detienes en un lugar indicado y te bajas del coche. Basta con pulsar un botón en la aplicación del \textit{smartphone} para que automáticamente el coche busque una plaza de aparcamiento entre las que están libres dentro del edificio del parking.\\

Estas son algunas de las tecnologías que emplean los coches autónomos o parcialmente autónomos en aparcamiento. Con estos sensores los vehículos son capaces de realizar maniobras complicadas, que a las personas nos resultan más tediosas. Las tecnologías avanzan rápido, y en este campo han avanzado mucho. Ya existen coches como el de Google que son autónomos por completo, aunque no se hayan comercializado aún por motivos legales. 

\subsection{Solución reactiva}
En esta práctica hemos desarrollado una solución ``ad hoc'' que resuelve el problema del aparcamiento autónomo. La solución reactiva se programa en el fichero \textit{MyAlgorithm.py}, en el método \textit{``execute''}, que se ejecuta periódicamente. Esto permite que la práctica se ejecute como un control reactivo, donde el coche recoge los datos de los sensores en cada instante y toma decisiones de movimiento basándose en estos datos. En esta práctica no se realiza ningún tipo de planificación antes del pilotaje, sino que se realiza el pilotaje directamente.\\


En la solución que se ha desarrollado el coche obtiene los datos que le proporciona cada láser en primer lugar. Para obtener estos datos, se emplea la función \textit{laser.getLaserData()} para cada correspondiente láser. Esta función nos devuelve los datos crudos del sensor láser, los cuales consisten en 180 pares de valores (distancia y ángulo). El siguiente paso ha sido hacer una media de los 180 valores del láser para tener conocimiento de cuál es la situación global del coche. Por la disposición espacial de los tres sensores se cubre casi los 360º alrededor del vehículo.\\

Al comienzo de la ejecución de la práctica el coche se encontrará estacionado más atrás de la plaza de aparcamiento, como se puede ver en la Figura~\ref{fig.Posicion1}. Los datos del láser son comprobados en cada iteración. El coche avanza a velocidad constante si no ha encontrado aún ninguna plaza de aparcamiento libre. Comprueba la distancia del láser de la derecha y el láser del parachoques trasero para ver si puede parar y estacionar el vehículo. Si los dos láseres tienen un valor entre un cierto rango de valores significa que el coche ha encontrado una plaza de aparcamiento libre. El coche parará en paralelo al coche de delante de la plaza de aparcamiento un momento antes de comenzar a aparcar. Podemos ver en la siguiente imagen cómo el coche alcanza el vehículo de delante de la plaza y para.


\begin{figure}[H]
  \begin{center}
    \includegraphics[width=0.7\textwidth]{figures/Autopark/Posicion1.png}
		\caption{Posición inicial del taxi en la práctica}
		\label{fig.Posicion1}
		\end{center}
\end{figure}


\begin{figure}[H]
  \begin{center}
    \includegraphics[width=0.7\textwidth]{figures/Autopark/Posicion2.png}
		\caption{Posición al parar en paralelo al coche de delante de la plaza libre}
		\label{fig.Posicion2}
		\end{center}
\end{figure}

Cuando el coche ya está parado en paralelo al vehículo de delante de la plaza de aparcamiento, entonces comienza la maniobra. Al principio el coche comenzará a dar marcha atrás lentamente, es decir, con una velocidad de tracción determinada. Pero también tendrá cierta velocidad de giro, con lo que describirá una especie de arco al aparcar (como lo hace un coche real al comenzar a aparcar en línea). El giro será hacia la derecha hasta que tenga una determinada inclinación. Esta maniobra la podemos ver en la Figura~\ref{fig.Posicion3}.

\begin{figure}[H]
  \begin{center}
    \includegraphics[width=0.7\textwidth]{figures/Autopark/Posicion3.png}
		\caption{Posición al dar marcha atrás con giro a la derecha}
		\label{fig.Posicion3}
		\end{center}
\end{figure}

Cuando el coche alcanza una determinada orientación, entonces se realiza otra maniobra para enderezar el vehículo. Ahora que ya ha entrado parte del coche en la plaza de aparcamiento, dará marcha atrás y realizará otra especie de arco. Es decir, que en esta ocasión el coche tendrá una velocidad de giro, pero hacia la izquierda. Esta maniobra se puede ver en la Figura~\ref{fig.Posicion4}.

\begin{figure}[H]
  \begin{center}
    \includegraphics[width=0.7\textwidth]{figures/Autopark/Posicion4.png}
		\caption{Posición al dar marcha atrás con giro a la izquierda}
		\label{fig.Posicion4}
		\end{center}
\end{figure}

El coche comprobará los datos del láser trasero en todo momento para saber si está cerca del coche de atrás. Seguirá marcha atrás y girando hacia a la izquierda hasta que detecte que está demasiado cerca del coche de atrás. Cuando detecte esta situación, entonces parará y comenzará a rectificar. Para realizar la rectificación dará marcha hacia delante muy despacio y girando hacia la derecha. Esta maniobra la realiza hasta quedar recto, es decir, de forma paralela a la acera. Cuando esté paralelo a la acera, el coche comprobará con los sensores láser frontal y trasero si se ha quedado muy pegado a un coche u otro. Si se ha quedado muy cerca del coche de atrás y tiene mucho margen por delante, entonces dará marcha hacia delante hasta quedar más o menos centrado en la plaza de aparcamiento. Si estuviera muy pegado al coche de delante, daría marcha atrás hasta estar centrado. Cuando el coche detecte que se ha quedado centrado en la plaza de aparcamiento, entonces parará. En la Figura~\ref{fig.Posicion5}, se puede observar cómo ha quedado perfectamente estacionado.

\begin{figure}[H]
  \begin{center}
    \includegraphics[width=0.7\textwidth]{figures/Autopark/Posicion5.png}
		\caption{Coche estacionado}
		\label{fig.Posicion5}
		\end{center}
\end{figure}


\section{Evaluador automático}
Se ha desarrollado también un evaluador automático que permite calificar objetivamente la solución a la práctica teniendo en cuenta diferentes parámetros. El evaluador automático muestra estos parámetros en una interfaz gráfica, así como la nota que obtiene el alumno. El evaluador automático se ha creado empleando PyQt5. Como sucedía en las anteriores prácticas, se ha programado mediante clases, que serán instanciadas en una clase principal. En la Figura~\ref{fig.Referee_DurantePractica} se muestra el evaluador automático durante la ejecución de la práctica.

\begin{figure}[H]
  \begin{center}
    \includegraphics[width=0.6\textwidth]{figures/Autopark/Referee_DurantePractica.png}
		\caption{Evaluador Automático del Autopark durante la ejecución de la práctica}
		\label{fig.Referee_DurantePractica}
		\end{center}
\end{figure}


En la esquina superior izquierda de su interfaz gráfico (Figura~\ref{fig.Referee_DurantePractica}), hay un visor que muestra un reloj digital. Este reloj va mostrando los segundos que han pasado desde que se ha ejecutado la práctica.\\

En la esquina inferior izquierda de la interfaz gráfica se representa parte de un círculo formado por diferentes colores, donde hay una aguja. Este semicírculo representa la orientación del robot, en función de determinados colores. Esto permite ver si nuestro vehículo está correctamente alineado o si por el contrario está algo desviado. Para que el coche esté perfectamente alineado tiene que tener una orientación de 0 grados. El color verde representa una orientación entre -15 y 15 grados. La zona anaranjada representa el rango de orientación entre -45 y -15 grados; y, entre 15 y 45 grados. El color rojo representa una orientación entre -45 y 45 grados. La aguja apuntará a un color u otro, y a cierta zona en función de la orientación que posea el robot.\\

En la esquina superior derecha se muestra en el visor diferentes distancias numéricas. La distancia a la acera que está pegada a la zona de aparcamiento, la distancia al coche de delante del hueco de aparcamiento, y la distancia al coche de atrás del hueco de aparcamiento. Esto permite comprobar el espacio libre delante y detrás y podemos ver si nos estamos acercando mucho a alguno de estos coches al realizar el aparcamiento, lo que podría ser peligroso.\\

Debajo de los mensajes de distancias el visor tiene una barra de progreso. Esta barra (de color rojo) aumentará en caso de que nos choquemos con algún coche al intentar aparcar. \\


En la parte central superior de la interfaz hay un botón, el cual se debe pulsar para saber la puntuación obtenida. Este botón (\textit{``Show me my mark''}) mostrará un mensaje con la nota conseguida. Esta nota se calcula en base al tiempo que se tarda en aparcar, la distancia a la que nos encontramos del hueco por si el taxi se ha quedado algo desviado, la orientación del taxi por si acaso se ha quedado torcido; y los choques que hayamos tenido con algún vehículo. 



\section{Experimentación}

\subsection{Ejecución típica}
La práctica “Aparcamiento automático” se puede ejecutar abriendo tres terminales y escribiendo lo siguiente en cada uno de ellos:

\begin{enumerate}[1.]
\item Lanzar Gazebo: gazebo autopark.world
\end{enumerate}
\begin{enumerate}[1b.]
\item Si el ordenador que se emplea no posee muchos recursos se puede arrancar el simulador sin interfaz gráfico: gzserver autopark.world
\end{enumerate}
\begin{enumerate}[2.]
\item	Ejecutar la práctica y lanzar la interfaz gráfica (\acrshort{gui}): python2 autopark.py -- --Ice.Config=autopark.cfg
\end{enumerate}
\begin{enumerate}[3.]
\item	Ejecutar el evaluador automático: python2 referee.py -- --Ice.Config=autopark.cfg
\end{enumerate}

En la Figura~\ref{fig.AparcamientoSecuencia} podemos ver una secuencia de imágenes donde empleando el algoritmo de la solución de referencia el coche autónomo es capaz de aparcar en línea correctamente. Una ejecución típica se puede ver en este video \footnote{\url{https://www.youtube.com/watch?v=2SYEb3DyWEE}}. 

\begin{figure}[H]
  \begin{center}
    \includegraphics[width=0.7\textwidth]{figures/Autopark/AparcamientoSecuencia.png}
		\caption{Secuencia de la ejecución típica de la solución de referencia}
		\label{fig.AparcamientoSecuencia}
		\end{center}
\end{figure}



\subsection{Aparcamiento lejano}

En este experimento hemos puesto inicialmente el coche más alejado de la plaza de aparcamiento para ver si era capaz de conducir hasta encontrarla y aparcar. El coche lo hemos desplazado en Gazebo 8 metros más atrás de la calle. En las pruebas hemos podido observar que el vehículo era capaz de hacer más recorrido conduciendo hacia delante hasta que encontraba la plaza de aparcamiento y estacionaba de forma correcta. A continuación, podemos ver una secuencia de imágenes, donde se puede apreciar desde qué posición iniciaba el pilotaje y cómo era capaz de aparcar.

\begin{figure}[H]
  \begin{center}
    \includegraphics[width=0.7\textwidth]{figures/Autopark/Experimento1.png}
		\caption{Coche en el aparcamiento lejano}
		\label{fig.Experimento1}
		\end{center}
\end{figure}


\subsection{Experimento con la plaza casi superada}
En la siguiente prueba se ha desplazado la posición inicial del vehículo hasta tenerlo en paralelo al coche que está delante de la plaza de aparcamiento. En esta prueba se ha comprobado que el coche no avanza hacia delante para buscar una plaza libre, sino que detecta que hay una plaza libre justo detrás de él y comienza a realizar la maniobra de aparcamiento. Este era el objetivo que se buscaba en esta prueba y el coche lo ha conseguido con éxito. En la Figura~\ref{fig.Experimento2} tenemos una secuencia de imágenes que muestran el coche al inicio y las maniobras realizadas por el mismo para aparcar.

\begin{figure}[H]
  \begin{center}
    \includegraphics[width=0.6\textwidth]{figures/Autopark/Experimento2.png}
		\caption{Aparcamiento con la plaza casi superada}
		\label{fig.Experimento2}
		\end{center}
\end{figure}
