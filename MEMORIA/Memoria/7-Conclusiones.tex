\chapter{Conclusiones}\label{cap.conclusiones}
En este capítulo se exponen las conclusiones finales obtenidas, así como los posibles trabajos futuros, tras analizar las tres prácticas aportadas (su infraestructura, aplicación académica, solución de referencia y evaluador automático de cada práctica) y algunos experimentos relevantes con las mismas.

\section{Conclusiones}
En este proyecto se ha alcanzado el objetivo global de crear tres prácticas académicas destinadas a la docencia en robótica en el entorno JdeRobot-Academy, extendiendo el repertorio de prácticas existente previamente. El propósito es que los alumnos puedan afianzar los conocimientos teóricos programando algunas técnicas robóticas, poniendo estos conocimientos en práctica. Además de la infraestructura y el software de apoyo, se ha desarrollado una solución de referencia por cada una de estas prácticas, que se ha validado experimentalmente.\\

En este proyecto se ha extendido el entorno de las prácticas partiendo de la idea de resolver en cada componente académico todos los problemas secundarios, pero que es necesario abordarlos para que los alumnos puedan programar la solución del algoritmo de cada práctica. Es decir, se abstrae a los alumnos de los problemas complejos que conlleva la práctica, tales como la comunicación con Gazebo, con los sensores del robot o incluso la calificación de la práctica que realiza el evaluador automático.\\

En primer lugar, se ha mejorado la versión anterior de la práctica ``TeleTaxi'': fallos corregidos, coche nuevo, solución de referencia nueva y evaluador automático. Asimismo, se ha propuesto una solución empleando el algoritmo \textit{Gradient Path Planning}, como se comentó en el capítulo~\ref{cap.gpp}. En esta solución se ha realizado una planificación global con el algoritmo GPP que sirve para que el robot pueda realizar el pilotaje correctamente. Esta planificación se ha programado mediante una estructura que expande el gradiente desde el punto de destino deseado hasta un poco más allá del punto donde se encuentra el robot. Esta planificación dará lugar a una rejilla con información del campo calculado, lo que ayudará al robot a navegar correctamente hasta el destino. Este campo ha sido calculado de tal forma que el robot no intente navegar pegado a los obstáculos y, por tanto, choque con ellos. Esta técnica conlleva cierta dificultad en el pilotaje del robot, ya que el robot no sigue una trayectoria determinada. El robot en el pilotaje evalúa en cada momento cuál es la mejor opción. Esto produce ciertas oscilaciones y afecta a la velocidad que debe llevar el robot. A pesar de estas dificultades, se han logrado unos resultados buenos como se describieron en la sección~\ref{cap.gpp}. \\

En los experimentos se ha apreciado la importancia que tiene el ordenador en el que ejecutemos el código, puesto que si es un ordenador sin muchas capacidades el robot irá más lento, ya que Gazebo consume muchos recursos. Esto se ha demostrado comparando el tiempo real y el tiempo simulado, como ya vimos en el capítulo~\ref{cap.gpp}.\\

El segundo subobjetivo era la creación de una nueva práctica llamada ``Aspiradora Autónoma'', lo cual se ha resuelto: creación de infraestructura, componente académico, solución de referencia y evaluador automático. Se ha propuesto una solución de referencia que no necesita que la aspiradora tenga autolocalización. Este algoritmo se inspira en el que emplean los modelos 500, 600, 700 u 800 de Roomba. En primer lugar, realizaba un patrón en espiral, a continuación, recorría el perímetro de la casa durante un cierto periodo de tiempo, y posteriormente hacía un algoritmo de ``cruce de habitación''. Hemos podido comprobar con los resultados que el tiempo en el estado ``recorrer perímetro'' influía en los resultados conseguidos. Es importante mencionar que el algoritmo de ``cruce de habitación'' consistía en un algoritmo aleatorio, donde el robot se movía hacia delante hasta chocar con un obstáculo, giraba hasta conseguir un ángulo aleatorio, y cuando conseguía este ángulo se movía hacia delante. De nuevo cabe destacar que, al ser aleatorio, los resultados obtenidos pueden variar bastante, por lo que puede que unas veces la aspiradora sea capaz de recorrer un gran porcentaje de la casa, mientras que otras veces no. Además, conforme vaya pasando el tiempo resulta más difícil que la aspiradora recorra zonas que no haya visitado ya.\\

Para alcanzar el tercer subobjetivo se ha creado la práctica ``Aparcamiento Automático'': creación de su infraestructura, componente académico, solución de referencia y evaluador automático. Se ha programado una solución de referencia empleando un algoritmo ``ad hoc'' en el que se usan los datos de los sensores láser para aparcar el robot correctamente. Vimos en el capítulo~\ref{cap.autoparking} cómo el vehículo busca una plaza libre de aparcamiento en base a los datos sensoriales ofrecidos por los tres láser. Gracias a estos datos era capaz de frenar justo en paralelo al coche de delante de la plaza libre y realizar la maniobra de aparcamiento. Esta maniobra la hacía hasta tener el coche perfectamente paralelo a la acera y haber dejado más o menos el mismo espacio delante y detrás del vehículo. En los experimentos que se han realizado hemos podido comprobar que era capaz de aparcar correctamente. El aparcamiento autónomo de un vehículo conlleva tener en cuenta demasiadas variables, lo que dificulta el algoritmo.\\

Se han satisfecho también otros requisitos que están implícitos en cada práctica, como emplear simuladores. Se ha trabajado sobre el simulador Gazebo, donde podíamos saber dónde estaba el robot en base a los sensores de posición. En los mundos virtuales para cada práctica se han simulado diferentes objetos, que suponen obstáculos para el robot.\\

En cuanto a los aportes personales, hemos aprendido a utilizar la plataforma JdeRobot para programar el comportamiento de diferentes robots autónomos. Uno de los elementos fundamentales de aprendizaje de esta plataforma es cómo se comunican los robots con los sensores y actuadores que poseen. \\

Este proyecto ha servido además para comprender las diferentes fases en las que se divide un trabajo de esta envergadura. Gracias a ello se ha aprendido a dividir un gran objetivo en pequeños objetivos de ingeniería, haciendo más fácil la solución de los mismos. No obstante, han surgido problemas durante el proyecto, más sencillos o más complejos, los cuales ha habido que solventar bien mediante más pruebas y experimentos, bien cambiando la técnica que se estaba empleando, o bien refinando el algoritmo que se empleaba hasta obtener los objetivos deseados.

\section{Trabajos futuros}
Debido a que este es un Trabajo de Fin de Grado no ha sido posible alargarlo ad infinitum para realizar más mejoras sobre el mismo. A continuación, se describirán posibles líneas concretas en las que se puede mejorar cada una de las prácticas.\\

Varias posibles mejoras que se podrían realizar en un futuro sobre la práctica ``TeleTaxi'' son:

\begin{itemize}
\item En este proyecto se ha resuelto el problema de la planificación mediante la técnica \textit{Gradient Path Planning}. En un futuro se podrían emplear otras técnicas de planificación, como un grafo de visibilidad, para comparar los resultados obtenidos con cada técnica y llevar a cabo un estudio más completo del problema.
\item Una dificultad incorporada en la navegación del robot era que el robot iba mirando en cada iteración sólo cuál era su posición respecto al mapa y el campo del gradiente calculado. Esto conllevaba que el robot pudiera navegar demasiado cerca de obstáculos o incluso chocar. Una posible mejora sería incorporar sensores en el robot, como sensores láser o cámaras, para detectar obstáculos y poder navegar con mayor precaución.
\item En este proyecto se ha validado el algoritmo únicamente sobre un simulador. Una posible mejora sería llevar el algoritmo propuesto a un robot real. Para ello es necesario conocer la posición del robot respecto al lugar por donde se mueve. Por lo que antes habría que dotarle de un algoritmo de autolocalización.
\end{itemize}

En la práctica ``Aspiradora Autónoma'', varias posibles mejoras que se podrían llevar a cabo son:

\begin{itemize}
\item La aspiradora se utiliza sin autolocalización, lo que limita el algoritmo, puesto que no tiene conocimientos previos de la casa antes de la navegación. Una posible mejora sería emplear el algoritmo \acrshort{slam} para crear un mapa de la casa. Esta información dotaría al robot de un mayor conocimiento de la casa, lo que haría posible que se lleve a cabo un algoritmo planificado, por el cual se puede recorrer la casa en menor tiempo.\\
\item Se podría realizar pruebas con una aspiradora real que incorporara el algoritmo desarrollado para comprobar qué resultados se obtendrían en diferentes habitaciones o incluso en una casa.
\end{itemize}

Varias posibles mejoras de la práctica ``Aparcamiento Automático'' son:

\begin{itemize}
\item El robot empleado en esta práctica únicamente poseía tres sensores láser, lo que limita el conocimiento de los alrededores del robot. Se podría añadir más sensores al robot, ya sean más sensores láser o cámaras, para dotarle de mayor precisión en el conocimiento de su entorno, lo que haría que se puediera llevar a cabo un algoritmo más seguro ante imprevistos.
\item El algoritmo que se ha desarrollado es un algoritmo reactivo basado directamente en los datos sensoriales. Otra posibilidad sería conocer el mapa del entorno por el que se mueve el robot y desarrollar algún tipo de planificación. Una posibilidad sería emplear el algoritmo \acrfull{rrt}, que es un método de planificación de trayectorias; o emplear la librería \acrfull{ompl}, que es un software que planifica trayectorias utilizando algoritmos basados en el mustreo. En estos ejemplos se podría tener en cuenta que se emplea un robot no holonómico.
\item Al igual que en las prácticas anteriores, esta práctica se ha validado sobre el simulador Gazebo. Una mejora sería ver cómo se comporta un robot real dotado con tres sensores láser al intentar aparcar en una plaza de aparcamiento, al igual que lo hacía nuestro robot simulado. Sería un experimento interesante a realizar para comprobar la generalidad de la solución desarrollada.
\end{itemize}

