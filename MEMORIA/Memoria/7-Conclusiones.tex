\chapter{Conclusiones y trabajos futuros}\label{cap.conclusiones}
En este capítulo se exponen las conclusiones obtenidas, así como los posibles trabajos futuros, tras analizar las tres prácticas (infraestructura, aplicación, árbitro y solución de cada práctica) y algunos experimentos relevantes de las mismas.

\section{Conclusiones}
Durante toda esta memoria se ha abordado la creación de tres prácticas académicas destinadas a la docencia en robótica. El propósito de la creación de estas prácticas es que los alumnos puedan afianzar los conocimientos teóricos de algunas técnicas robóticas, poniendo estos conocimientos en práctica sobre este entorno docente. Además, se ha propuesto una solución por cada una de estas prácticas, así como la experimentación con cada una de ellas.\\

En este proyecto se ha desarrollado el entorno de las prácticas partiendo de la idea de resolver todos los problemas que son ajenos a los alumnos, pero que al mismo tiempo es necesario abordarlos para que los alumnos puedan programar la solución del algoritmo de cada práctica y puedan ejecutar sus algoritmos. Es decir, se abstrae a los alumnos de los problemas complejos que conlleva la práctica, tales como la comunicación del entorno en Gazebo con la aplicación, la comunicación de la aplicación con los sensores del robot o incluso la calificación de la práctica que realiza el evaluador automático.\\

En la práctica ``TeleTaxi'' se ha propuesto una solución empleando el algoritmo Gradient Path Planning, como se comentó en el capítulo 4. En esta solución se ha realizado una planificación global con el algoritmo GPP que servirá posteriormente para que el robot pueda realizar el pilotaje correctamente. Esta planificación se ha programado mediante una estructura que expande el gradiente desde el punto de destino deseado hasta un poco más allá del punto donde se encuentra el robot. Esta planificación dará lugar a una rejilla con información del campo calculado, lo que ayudará al robot a navegar correctamente hasta el destino. Este campo ha sido calculado de tal forma que el robot no intente navegar pegado a los obstáculos y, por tanto, choque con ellos. Esta técnica conllevará una gran dificultad en el pilotaje del robot, ya que el robot no sigue una trayectoria determinada. El robot en el pilotaje evalúa en cada momento cuál es la mejor opción. Esto produce ciertas oscilaciones y afecta a la velocidad que debe llevar el robot. A pesar de estas dificultades, se han logrado unos resultados buenos. \\

En esta práctica se ha apreciado la importancia que tiene el ordenador en el que ejecutemos la práctica, puesto que si es un ordenador sin muchas capacidades el robot irá más lento, ya que Gazebo consume muchos recursos. Esto se ha demostrado comparando el tiempo real y el tiempo simulado, como ya vimos en el capítulo~\ref{cap.gpp}.\\

En la práctica ``Aspiradora Autónoma'' se ha propuesto una solución que consiste en un algoritmo sin autolocalización. En el capítulo~\ref{cap.roomba} hemos visto que este algoritmo se basaba en el algoritmo que emplean los modelos 500, 600, 700 u 800 de Roomba. En primer lugar, realizaba un patrón en espiral, a continuación, recorría el perímetro de la casa durante un cierto periodo de tiempo, y posteriormente hacía un algoritmo de ``cruce de habitación''. Hemos podido comprobar con los resultados que el tiempo que Roomba recorría el perímetro influía en los resultados conseguidos. Es importante mencionar que el algoritmo de ``cruce de habitación'' consistía en un algoritmo aleatorio, donde el robot hacía una especie de choca-gira aleatorio. Es decir, el robot se movía hacia delante hasta chocar con un obstáculo. En este caso la aspiradora giraba con sentido aleatorio hasta conseguir girar un ángulo aleatorio, y cuando conseguía este ángulo se movía hacia delante. Cabe destacar que, al realizar un algoritmo aleatorio, los resultados obtenidos pueden variar bastante, por lo que puede que unas veces la aspiradora sea capaz de recorrer un gran porcentaje de la casa, mientras que otras veces no. Además, conforme vaya pasando el tiempo resulta más difícil que la aspiradora recorra zonas que no haya visitado ya.\\

En la práctica ``Aparcamiento Automático'' se ha llevado a cabo una solución empleando un algoritmo ``ad hoc'' en el que se usan los datos de los sensores láser para poder aparcar el robot correctamente. En esta solución vimos en el capítulo~\ref{cap.autoparking} cómo el vehículo buscaba una plaza libre de aparcamiento en base a los datos sensoriales ofrecidos por cada láser. Gracias a estos datos era capaz de frenar justo en paralelo al coche de delante de la plaza libre y realizar la maniobra de aparcamiento. Esta maniobra la hacía hasta tener el coche perfectamente paralelo a la acera y haber dejado más o menos el mismo espacio delante y detrás del vehículo. Es decir, se pudo ver cómo el coche quedaba perfectamente aparcado. En los experimentos que se han realizado hemos podido comprobar que el taxi era capaz de aparcar correctamente. Se ha podido comprobar que el aparcamiento autónomo de un vehículo conlleva tener en cuenta demasiadas variables, lo que dificulta el algoritmo.\\

A lo largo de la memoria podemos ver que se llevan a cabo otros requisitos que están implícitos en cada práctica, como trabajar con simuladores. Para ello se ha trabajado sobre el simulador Gazebo, donde podíamos saber dónde estaba el robot en base a los sensores de posición. En este simulador se han simulado diferentes objetos, que suponen obstáculos para el robot.\\

En este proyecto hemos aprendido a utilizar la plataforma JdeRobot para desarrollar el comportamiento de diferentes robots autónomos. Uno de los elementos fundamentales de aprendizaje de esta plataforma es cómo se comunican los robots con los sensores y actuadores que poseen. \\

Este proyecto ha servido para comprender las diferentes fases en las que se divide un trabajo de esta envergadura. Gracias a ello se ha aprendido a dividir un gran objetivo en pequeños objetivos que tenemos que conseguir, haciendo más fácil la solución de los mismos. No obstante, han surgido problemas durante el proyecto, más sencillos o más complejos, los cuales ha habido que solventar bien mediante más pruebas y experimentos, bien cambiando la técnica que se estaba empleando, o bien refinando el algoritmo que se empleaba hasta obtener los objetivos deseados.

\section{Trabajos futuros}
Debido a que este es un proyecto de fin de Grado no ha sido posible alargarlo para realizar mejoras sobre el mismo. A continuación, se describirán posibles mejoras para cada una de las prácticas.\\

Las posibles mejoras que se podrían realizar en un futuro sobre la práctica ``TeleTaxi'' son las siguientes:

\begin{itemize}
\item En este proyecto se ha resuelto el problema de la planificación mediante la técnica Gradient Path Planning. En un futuro se podrían emplear otras técnicas de planificación, como la creación de un grafo de visibilidad, para poder comparar los resultados obtenidos con cada técnica y llevar a cabo un estudio más completo del problema.
\item Una dificultad incorporada en la navegación del robot era que el robot iba mirando en cada iteración cuál era su situación respecto al mapa con el campo del gradiente calculado. Esto conllevaba que el robot pudiera navegar cerca de obstáculos o incluso chocar. Una posible mejora sería incorporar sensores en el robot, como por ejemplo sensores láser o cámaras, para poder detectar obstáculos y poder navegar con mayor precaución.
\item En este proyecto se ha realizado el algoritmo únicamente sobre un simulador. Una posible mejora sería llevar el algoritmo propuesto a un robot real. Pero existe aún un problema en esta práctica, ya que es necesario conocer la posición del robot respecto al lugar por donde se mueve. Por lo que antes habría que resolver este inconveniente.
\end{itemize}

En la práctica ``Aspiradora Autónoma'', las posibles mejoras que se podrían llevar a cabo son las siguientes:

\begin{itemize}
\item La aspiradora que se utiliza es sin autolocalización, lo que limita el algoritmo, puesto que no tiene conocimientos previos de la casa antes de la navegación. Una posible mejora sería emplear el algoritmo \acrshort{slam} para poder crear un mapa de la casa. Esta información dota al robot de un mayor conocimiento de la casa, lo que hace posible que se lleve a cabo un algoritmo planificado, por el cual se puede recorrer la casa en menor tiempo.\\
\item En esta práctica se ha trabajado sobre un simulador, al igual que en la anterior. Se podría realizar pruebas con una aspiradora real que incorporará el algoritmo desarrollado para comprobar qué resultados se obtendrían en diferentes habitaciones o incluso en una casa.
\end{itemize}

Las posibles mejoras de la práctica ``Aparcamiento Automático'' se describen a continuación:

\begin{itemize}
\item El robot empleado en esta práctica únicamente poseía tres sensores láser, lo que limita el conocimiento de los alrededores del robot. Se podría añadir más sensores al robot, ya sean más sensores láser o cámaras, para dotarle de mayor precisión en el conocimiento de su entorno, lo que hace que se pueda llevar a cabo un algoritmo más seguro ante imprevistos.
\item El algoritmo que se ha desarrollado es un algoritmo reactivo basado en los datos sensoriales. Otra posibilidad sería conocer el mapa del entorno por el que se mueve el robot y desarrollar algún tipo de planificación. Una posibilidad sería emplear el algoritmo \acrfull{rrt}, que es un método de planificación de trayectorias. Con este método se puede tener en cuenta que se emplea un robot no holonómico.
\item Al igual que en las prácticas anteriores, esta práctica se desarrolla sobre el simulador Gazebo. Una mejora sería ver cómo se comporta un robot dotado con tres sensores láser al intentar aparcar en una plaza de aparcamiento, al igual que lo hacía nuestro robot simulado en la práctica. Sería un gran experimento a realizar para comprobar los resultados obtenidos.
\end{itemize}

