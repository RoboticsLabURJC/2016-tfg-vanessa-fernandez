\chapter*{Resumen}

En los últimos años, la presencia de la robótica ha aumentado considerablemente, haciendo que cada vez sea más necesario tener una formación en este ámbito. El entorno docente JdeRobot-Academy permite acercar esta materia a los alumnos de forma sencilla y eficaz. Este entorno consta de diferentes prácticas para la enseñanza de técnicas robóticas. En este proyecto se ha tratado de mejorar el abanico de posibilidades que ofrece este entorno, ampliando y mejorando las prácticas ya existentes.\\

Para lograr este fin se han empleado múltiples herramientas. Creando nuevos mundos en el simulador Gazebo, creando los componentes académicos necesarios, e incluso creando un evaluador automático de prácticas. El objetivo de estos componentes académicos es abstraer a los alumnos de todas las complejidades de la interfaz gráfica, y conexiones con los sensores y actuadores; haciendo posible que el alumno se centre en la resolución de los algoritmos que se plantean en cada práctica.\\

Se pretende mejorar la práctica ``TeleTaxi'' (creada con el fin de resolver un algoritmo de navegación global) para reproducir un entorno más realista, y mejorar el componente académico que sirve de apoyo al alumno. Además, se ha dotado a la práctica de un evaluador automático que permite calificar el algoritmo que programa el alumno. Asimismo, se propondrá una solución para esta práctica, donde se empleará la técnica Gradient Path Planning. Dicha técnica se emplea para guiar al robot desde su posición inicial hasta la posición de destino, evitando chocar con los obstáculos descritos en el mapa.\\

Con el propósito de mejorar el entorno docente, se creará la práctica ``Aspiradora autónoma''. La elaboración de la misma incluirá la creación de la infraestructura necesaria, el componente académico y el evaluador automático. Esta práctica pretende que el alumno se familiarice con un algoritmo de navegación sin autolocalización, donde el propósito es que el robot recorra la mayor superficie posible de una casa. Este algoritmo se basará en los algoritmos de navegación que emplean los modelos de la serie 500 de Roomba de iRobot. En esta práctica se ha desarrollado una solución que resuelve el problema planteado.\\

Con el mismo fin se creará la práctica ``Aparcamiento automático'', la cual implicará el desarrollo de la infraestructura necesaria para crear un entorno apropiado, la creación del componente académico que simplifica al alumno la resolución de la práctica, y la creación de un evaluador automático. El propósito de esta práctica es que el alumno tome contacto con técnicas de aparcamiento autónomo. En el proyecto se desarrollará una solución “ad hoc”, que tiene en cuenta los datos que proporcionan los sensores para poder tomar decisiones y aparcar de forma correcta.