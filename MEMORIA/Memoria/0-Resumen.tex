\chapter*{Resumen}

En los últimos años, la presencia de la robótica ha aumentado considerablemente, haciendo que cada vez sea más necesario tener una formación en este ámbito. El entorno docente JdeRobot-Academy permite acercar esta materia a los alumnos universitarios de forma sencilla y eficaz. Este entorno consta de diferentes prácticas para la enseñanza de técnicas robóticas. En este proyecto se ha extendido el abanico de posibilidades que ofrece este entorno, mejorando una práctica ya existente y creando otras dos.\\

Creando nuevos mundos en el simulador Gazebo, creando los componentes académicos necesarios, e incluso creando un evaluador automático de prácticas. El objetivo de estos componentes académicos es abstraer a los alumnos de todas las complejidades de la interfaz gráfica, y conexiones con los sensores y actuadores; haciendo posible que el alumno se centre en la resolución de los algoritmos.\\

Se ha mejorado la práctica ``TeleTaxi'', creada con el fin de resolver un algoritmo de navegación global. Además, se ha dotado a la práctica de un evaluador automático que permite calificar el algoritmo que programa el alumno. Asimismo, se propone una solución de referencia, donde se empleará la técnica \textit{Gradient Path Planning}.\\

Se ha creado la práctica ``Aspiradora autónoma'' para que el alumno se familiarice con un algoritmo de navegación sin autolocalización, donde el propósito es que el robot recorra la mayor superficie posible de una casa. Se basa en los algoritmos de navegación que emplean los modelos de la serie 500 de Roomba de iRobot. Se ha desarrollado una solución de referencia.\\

Se ha creado también la práctica ``Aparcamiento automático'', desarrollando la infraestructura necesaria en el simulador Gazebo, la creación del componente académico que simplifica al alumno la resolución de la práctica, y la creación de un evaluador automático. Se ha desarrollado también una solución “ad hoc” reactiva basada en estados, que tiene en cuenta datos sensoriales para tomar decisiones y aparcar de forma correcta.